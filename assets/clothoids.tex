\documentclass[12pt,a4paper]{scrartcl}

\usepackage[T1]{fontenc}
\usepackage[utf8]{inputenc}
\usepackage{amsmath}
\usepackage{graphicx}
\usepackage{float}

\floatplacement{figure}{H}

\title{Parallel curve of a Clothoid \thanks{Translated from the german original, found in the 1907 edition of the \emph{Archiv der Mathematik und Physik}}}
\author{H. Wieleitner}
\date{August 1, 1906}

\begin{document}

\maketitle
\begin{figure}
  \caption{Parallel curve of a clothoid}
  \centering
  \includegraphics[width=0.75\textwidth]{curve}
\end{figure}
The clothoid with the equation $\rho=a^2/s$, $\rho$ being the radius of curvature and $s$ the arc, has an inflection point $[s=\pm0, \rho=\pm\infty]$
in its starting point $O$.  Additionally it has two asymptotic points $M$ and $M'$ (Tangent angle $\varphi=s^2/2a^2$, which is $\infty$ for $s=\pm\infty;\rho=0$).
Their coordinates are $x=y=\pm\frac{a}{2}\sqrt{\pi}$.

Its parallel curve does not seem to have been examined yet.  For it
\begin{equation}
s'=s+l\varphi, \rho'=\rho+l
\end{equation}
applies, $l$ being the distance (Cesàro, \emph{Nat. G.}, §19. --- Loria, S. 645).
This immediately results in
\begin{displaymath}
s=\frac{a^2}{\rho'-l}, s'=s+\frac{s^2l}{2a^2}
\end{displaymath}
and by insertion
\begin{equation}
s'=\frac{a^2(2\rho'-l)}{2{(\rho'-l)}^2} \label{eq:s}
\end{equation}
or transposed
\begin{equation}
\rho'=\frac{a^2}{2s'}+l\pm\frac{a}{2s'}\sqrt{a^2+2ls'} \tag{2*}\label{eq:rho}
\end{equation}.  Further
\begin{equation}
\varphi'=\int{\frac{d s'}{\rho'}}=\int{\frac{\left(1+\frac{ls}{a^2}\right)d s}{\frac{a^2}{s}+l}}=\int{\frac{sds }{a^2}}=\varphi \label{eq:phi}
\end{equation}
as is obvious for a parallel curve.

From (\ref{eq:s}) and (\ref{eq:rho}), $s'$ is a distinct function of $\rho'$ while $\rho'$ is an ambiguous function of $s'$.
To better see this interdependency, drawing (\ref{eq:s}) into a right-angled coordinate-system (see figure~\ref{fig:small}) helps.

\begin{figure}
  \caption{Relation of $s'$ and $\rho'$}\label{fig:small}
  \centering
  \includegraphics[width=0.25\textwidth]{smallfig}
\end{figure}

For $s'=\pm0$, $\rho$ will be $\pm\infty$; this yields the inflection point in the starting point $O'$ of the arc.
If we turn right, $\rho'$ falls with rising $s'$, up to $l$ because for $s'=+\infty$, $\rho'$ equals $l$.
But, because of (\ref{eq:phi}), $\varphi'$ is also $\infty$.  This means, the curve has an asymptotic circle around $M$ with radius $l$, which it closes in on from the outside.
If we now go left from $O'$, ie we take $s'$ negative, then $\rho'$ goes from $-\infty$ to $0$ at $s'=-m=-a^2/2l^2$.
Because of (\ref{eq:rho}) this is also a minimum for $s'$.  Our curve has a sharp turn at this point, that we will call $S$, with the tangential direction $\varphi'=a^2/2l^2$.
The smaller $l$ is in relation to the clothoid, the more turns around $M'$ the branch going left from $O'$ will make until it reaches $S$.
From $S$ onwards $s'$ rises (as seen in figure~\ref{fig:small}) up to the point $O''$ for which $|O'S|=|SO''|$.
In $O''$, $s'$ is $0$ and $\rho'=\frac{1}{2}l$.  Apart from that, the point is not visible on the curve.
Continuing on, $s'$ and $\varphi'$ go to $+\infty$ while $\rho'$ tends to $l$.
This means, the curve has a second asymptotic circle around $M'$, which it closes in on from the inside.

Changing the sign of $l$ just switches $M$ and $M'$.

\end{document}
